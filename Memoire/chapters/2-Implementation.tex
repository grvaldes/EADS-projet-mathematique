Le problème du voyageur de commerce (TSP, par son nom en anglais) \cite{book:TSP,chap:TSP} est un problème très connu et étudié dans le domaine de l'informatique et l'optimisation. Il cherche à répondre la question: "étant donné $n$ villes et leurs distances, quel est le chemin le plus court qu'un voyageur peut faire pour passer pour tous les villes une seule fois eet revenir au point de départ".

Le TSP comme problème est connu au moins depuis le 19ème siècle \cite{book:TSP}, mais sa formalisation mathématique généralisé apparaît dans les années 1930 \cite{book:TSP} est popularisé avec ce nom par la RAND corporation en 1949 \cite{book:TSP}. À partir de cette publication, le problème a devenu très populaire dû à la simplicité de son énoncé en comparaison a son résolution. Il est utilisé comme référence pour évaluer l'efficience des algorithmes. % Find citations

Le TSP a diverses applications dans plusieurs domaines comme la logistique, séquençage d'ADN, manufacture, et autres. Chaque problème composé d'un ensemble que doit être rempli dans un ordre particulier pourrait être pensé comme le problème TSP, où on cherche un chemin moins coûteux. Le problème peut être résolu exactement si tous les chemins entre villes sont analyses et comparés, mais la quantité de chemins a vérifier augmente rapidement. Pour ce raison, des algorithmes sont proposés pour trouver des solutions approximatives que peuvent être optimales ou pas, mais qui résolut le problème rapidement. Dans ce travail, seulement la version symétrique de l'algorithme sera analyse. Cela veut dire que les distances consideres entre villes est égales dans les deux sens. Les villes sont aussi considérés toutes interconnectés, donc il existera toujours une connection directe entre une ville et l'autre.

\section{Algorithmes pour le résoudre}
Le problème du voyageur de commerce est peut-être le plus étudié dans la théorie de la complexité, donc plusieurs des algorithmes ont été proposés pour trouver une solution optimale. Ici, quatre algorithmes à analyser sont présentés.

\subsection{Nearest Neighbors}
Le voisin plus proche \cite{article:nearest} est un

\begin{algorithm}[H]
\DontPrintSemicolon
\KwIn{Matrice de distance (\texttt{distance[\,][\,]}), Liste de villes (\texttt{villes[\,]})}
\KwOut{Ordre de visite (\texttt{ordre\_visite}) et distance totale (\texttt{distance\_totale})}
\vspace{10pt}
\texttt{actuelle} $\gets$ indice ville départ\;
\texttt{visitee[\,]} $\gets$ liste de booleans False pour chaque ville\;
\texttt{ordre\_visite[0]} $\gets$ actuelle\;
\texttt{visitee[actuelle]} $\gets$ \texttt{True}\;
\texttt{distance\_totale} $\gets 0$\;

\vspace{10pt}
\While{il reste de villes à visiter}{
    \texttt{meilleur\_distance} $\gets \infty$\;
    \texttt{meilleur\_voisin} $\gets$ \texttt{None}\;
    
    \vspace{10pt}
    \ForEach{voisin $v$ et sa distance $d$ de la ville actuelle}{
        \If{voisin pas visité \textbf{et} distance < meilleur\_distance}{
            \texttt{meilleur\_distance} $\gets$ distance[actuelle][v]\;
            \texttt{meilleur\_voisin} $\gets v$\;
        }
    }

    \vspace{10pt}
    \If{il n'y a pas de meilleur voisin}{
        \textbf{break}
    }
    
    \vspace{10pt}
    \texttt{actuelle} $\gets$ indice du meilleur voisin\;
    \texttt{visitee[actuelle]} $\gets$ \texttt{True}\;
    \texttt{ordre\_visite}$\gets$ ajouter ville actuelle\;
    \texttt{distance\_totale} $\gets$ \texttt{distance\_totale} + \texttt{meilleur\_distance}\;
}

\vspace{10pt}
\ForEach{ville visitée $v$ en ordre inverse}{
  \If {distance avec ville de départ > 0}{
    \texttt{ordre\_visite}$\gets$ ajouter ville de départ\;
    \texttt{distance\_totale} $\gets$ \texttt{distance\_totale} + \texttt{distance[v][depart]}\;
    \texttt{actuelle} $\gets$ indice du ville de départ\;
  } \Else {
    \texttt{ordre\_visite}$\gets$ ajouter ville antérieure\;
    \texttt{distance\_totale} $\gets$ \texttt{distance\_totale} + \texttt{distance[v][v-1]}\;
    \texttt{actuelle} $\gets$ indice du ville de antérieure\;
  }
}

\vspace{10pt}
\Return \texttt{ordre\_visite}, \texttt{distance\_totale}\;
\caption{Algorithme pour les voisins plus proches}
\end{algorithm}

\subsection{Algorithme 2-opt}
L'algorithme 2-opt \cite{article:2opt}

\begin{algorithm}[H]
\DontPrintSemicolon
\KwIn{Initial tour from Nearest Neighbor}
\KwOut{Improved tour and total distance}

\vspace{10pt}
improved $\gets$ \texttt{true}\;
tour $\gets$ visit\_order excluding last city\;
best\_distance $\gets$ total distance of tour + return to start\;

\vspace{10pt}
\While{improved}{
    improved $\gets$ \texttt{false}\;
    
    \vspace{10pt}
    \For{$i \gets 1$ \KwTo $n - 2$}{
        \For{$j \gets i + 1$ \KwTo $n - 1$}{
            new\_tour $\gets$ tour with segment $[i:j]$ reversed\;
            new\_distance $\gets$ distance of new\_tour + return to start\;

            \If{new\_distance $<$ best\_distance}{
                tour $\gets$ new\_tour\;
                best\_distance $\gets$ new\_distance\;
                improved $\gets$ \texttt{true}\;
                \textbf{break}\;
            }
        }
        
        \vspace{10pt}
        \If{improved}{\textbf{break}}
    }
}

\vspace{10pt}
visit\_order $\gets$ tour + [start city]\;
total\_distance $\gets$ best\_distance\;

\vspace{10pt}
\Return visit\_order\;
\caption{2-opt Optimization}
\end{algorithm}

\subsection{Algorithme 3}
\subsection{Algorithme 4}